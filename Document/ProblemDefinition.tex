%!TEX root = Main.tex
\documentclass[Main]{subfiles}
\begin{document}
\chapter{Problem Definition}
The purpose of this case project is to take the learnings of the course Non-linear Signal Processing and Pattern Recognition (TINONS1) and apply them to a specific signal processing/recognition problem.
In this case the object is to create a speaker recognition system.
The system shall be able to recognize a specific speaker from a ensemble of speaker models.
Firstly the system should work with specific sentences, and later perhaps with ambiguous speech.

It is also a possibility that the system will be implemented to work as a speaker verification system.
Such a system would gauge the likelihood af a speaker matching a speaker model and make a decision as to the validity of the speakers identity.

The topics from the TINONS course that will be applied in relation to this case are:
\begin{itemize}
\item Linear regression 
\fxnote{Indsæt korrekt term til linear halløj} 

\item Dimensionality reduction

\item Gaussian Mixture Models

\item \fxnote{Indsæt flere løbende}

\cite{RefWorks:22}

\end{itemize}


\end{document}


