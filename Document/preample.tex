\documentclass[danish,a4paper,twocolumn,amsmath,amssymb]{revtex4-1}
\usepackage{babel}		%Giver mulighed for dansk orddeling. Slet kun hvis du VED hvad du laver, eller skal skrive noget på engelsk.
\usepackage[latin1]{inputenc}	%Tillader danske tegn
\usepackage[T1]{fontenc}	%Tillader danske tegn
\usepackage{graphicx}		%Tillader indsættelse af billeder
\usepackage{dcolumn}		%Bruges til at lave matematiske tabelsøjler... se datatabel
\usepackage{booktabs}		%linjer i tabeller...
\usepackage{mathtools}		%Ekstra matematik... bare lad den være, du får muligvis brug for den.
\usepackage{multirow}
\usepackage{threeparttable} 			% Jeg kan ikke huske hvad den gør, men den skal bruges til for at tabelnoterne virker
\usepackage[tableposition=top]{caption} % Noget med at lave caption så det står godt
\usepackage[version=3]{mhchem}  %\ce{H2O + e^{-} -> H2O^{+} + 2e^-}
%siunitx-pakken er ny ift. den originale template, så der henvises i tekstan til en anden pakke, beklager.
\usepackage{siunitx}	%Bruges \SI{<tal>}{<enhed>}, \si{<enhed>} eller \num{<tal>}.
\sisetup{output-decimal-marker={,},separate-uncertainty=true}%Sørger for komma som decimalmarkør. Virker også ved decimaltal, hvis man bruger \num{<tal>}.
\usepackage{url}		 %bruges til at formattere url'er... kan sagtens udelades.
%Det følgende laver to makroer, \tref{} og \fref, der kan bruges ligesom \ref til at referere til hhv. tabeller og figurer. 
%De indsætter selv ordet Tabel/Figur, og sørger for at der ikke sker et linjebrud mellem dette og nummeret.
\newcommand{\tref}[1]{\tablename~\ref{#1}}
\newcommand{\fref}[1]{\figurename~\ref{#1}}
%Tilsvarende for ligninger. Indsætter "ligning (#)".
\newcommand{\lref}[1]{ligning~\eqref{#1}}
	% \eqref laver en reference med parenteser omkring (til brug ved ligninger.)

%Disse makroer indsætter ordene "PicoScope" og "EasyPlot" i teksten (med store bogstaver. Husk at sætte "{}" bagefter for at få et mellemrum.
%Jeg har lavet dem fordi jeg blev træt af at sidde og trykke shift hele tiden, og for at få det til at stå ens. Brug dem, eller lad være.
\newcommand{\picos}[0]{\textsc{PicoScope}} %hedder \picos for ikke at komme i kambolage med pico fra SIunits.
\newcommand{\epw}[0]{\textsc{EasyPlot}}    %epw er navnet på programfilen for easyplot, men det har ingen betydning for makroen. Jeg valgte det fordi det var noget jeg kunne huske, og det kan sagtens ændres.
\newcommand{\matl}[0]{\textsc{Matlab}} %Skriver Matlab med small caps.
\usepackage[footnote,draft,english,silent,nomargin]{fixme}
%hyperref-pakken kan bruges til at redigere pdf-metadata. Det kan være et nice touch, men er generelt ikke påkrævet. Laver automatisk referencer i teksten til farvede hyperlinks i.
\usepackage{hyperref}
\hypersetup
{   pdfsubject={Rapport},
	pdfauthor={Alexander Rasborg Knudsen}
    pdftitle={Masse},
    pdfstartview=FitH,
    colorlinks=true}
    
%Følgende gør, at subscripts bliver ikke-kursiv. Anvendes X_|<subscript>|. Erstattes evt. med X_{\mathrm{<subscript>}}.
\makeatletter
\begingroup
\catcode`\_=\active
\protected\gdef_{\@ifnextchar|\subtextup\sb}
\endgroup
\def\subtextup|#1|{\sb{\textup{#1}}}
\AtBeginDocument{\catcode`\_=12 \mathcode`\_=32768 }
\makeatother

\usepackage[danish=quotes]{csquotes} %Danske citationstegn. \enquote{}

%Lad disse to linjer være. De sørger for at bunden af siden bliver pæn, og fjerner indryk ved afsnit.
\raggedbottom
\parindent = 0pt

% A file like this(called home.tex) could be placed in each latex project folder. 

% This will give the possibility to use modules like this:
% % A file like this(called home.tex) could be placed in each latex project folder. 

% This will give the possibility to use modules like this:
% % A file like this(called home.tex) could be placed in each latex project folder. 

% This will give the possibility to use modules like this:
% \input{home} - This is needed to use the \home command
% \input{\home/Modules/Usepackages}
% \input{\home/Modules/ChapterStyle}
% \input{\home/Modules/HeaderAndFooter}

% If you use Github or any other collaborating tool, you should ignore the home.tex file, so that every user could have their own home.tex file.

\newcommand{\home}{C:/Users/Alexander/Documents/GitHub/LaTeX} - This is needed to use the \home command
% \input{\home/Modules/Usepackages}
% \input{\home/Modules/ChapterStyle}
% \input{\home/Modules/HeaderAndFooter}

% If you use Github or any other collaborating tool, you should ignore the home.tex file, so that every user could have their own home.tex file.

\newcommand{\home}{C:/Users/Alexander/Documents/GitHub/LaTeX} - This is needed to use the \home command
% \input{\home/Modules/Usepackages}
% \input{\home/Modules/ChapterStyle}
% \input{\home/Modules/HeaderAndFooter}

% If you use Github or any other collaborating tool, you should ignore the home.tex file, so that every user could have their own home.tex file.

\newcommand{\home}{C:/Users/Alexander/Documents/GitHub/LaTeX}
%\input{\home/Modules/Usepackages}
%\input{\home/Modules/ChapterStyle}
%\input{\home/Modules/HeaderAndFooter}
%\input{\home/Modules/Paragraph}
\input{\home/Modules/Figure}