%!TEX root = Main.tex
\subsection*{Feature Extraction}
When doing speaker recognition it is advantageous to classify on the basis of extracted features from speech data, rather than the audio samples themselves \cite{Springer:36}.
Features were extracted on a frame-to-frame basis.
This was done on the assumption of pseudo-stationarity of human speech in the scale of a few tens of ms \cite{Springer:36}.
The frames consisted of 256 samples, with a new frame staring every 100 samples.
This means that we have a frame length of
\begin{equation}
t_{frame length} = \dfrac{N_{frame}}{F_s} = \dfrac{256}{48\ kHz} = 5.33\ ms
\end{equation}

and a frame interval of
\begin{equation}
t_{frame interval} = \dfrac{N_{interval}}{F_s} = \dfrac{100}{48\ kHz} = 2.08\ ms
\end{equation}

The features extracted for use in classification were $12^{th}$ order Mel-frequency Cepstral Coefficients (MFCC), as they hava proven effective in speaker recognition applications and speech processing in general.
MFCCs describe the the Mel cepstrum, a spectrum-of-a-spectrum on the non-linear Mel-scale.
The process for calculation MFCCs is described in figure \ref{fig:MFCC_Flowchart} below.

\begin{figure}[H]
\centering
\includegraphics[width=0.7\linewidth]{MFCC_Flowchart_CROP}
\caption{Process for calculating MFCC}
\label{fig:MFCC_Flowchart}
\end{figure}

Further, so-called delta and double-delta coefficients, the temporal derivatives $ \frac{dC}{dt} $ and double-derivatives $\frac{d^2C}{dt}$ of the MFCCs respectively, are used as features.
In this project the calculation of MFCCs, delta and double-delta coefficients is done using the free Voicebox toolbox for MATLAB \cite{voicebox}

For more on MFCC and feature extraction, see Appendix.
