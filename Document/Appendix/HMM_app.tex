\chapter{Hidden Markov Models}
In this project the Hidden Markov models (HMM) is not used on the data.
The reason for this is that the HMM is optimized for recognizing the sequence of the sounds which combines to a word.
HMM is focused on speech recognition and not speaker recognition, which is this project works with.

\section{Theory}   
The HMM is developed to process sequential information in the data, in other models the features have been treated as independent and identical distribution.
When classifying speech from a sequence of data recorded over time, the features tend to be strongly associate with the previous features.
The associated features can be exploited by the Markov model to classify, the model uses join distributions to link the relationship between two or more successive data points $ \left\lbrace \mathbf{x}_1,...,\mathbf{x}_N \right\rbrace  $.
\begin{equation}
p(\mathbf{x}_1,...,\mathbf{x}_N) = 
\prod_{n=2}^{N} 
p(\mathbf{x}_n | \mathbf{x}_1,...,\mathbf{x}_{n-1})
\label{eq:HMM_JD}
\end{equation} 
The right hand-side describes the probability of a transition from one data point to the next.

If the data is noisy, an addition to Markov model can be used, which is called Hidden Markov model.
The HMM uses discrete hidden states $ \left\lbrace \mathbf{z}_1,..., \mathbf{z}_N \right\rbrace  $.
The jointed distributions in the HMM can be formulated so
\begin{equation}
p(\mathbf{X},\mathbf{Z}|\Theta) = 
p(\mathbf{z}_1|\pi)
\left(\prod_{n=2}^{N} p(\mathbf{z}_n|\mathbf{z}_{n-1},\mathbf{A}) \right)
\prod_{m=1}^{N} p(\mathbf{x}_m|\mathbf{z}_m,\phi) 
\label{eq:HMM}
\end{equation}
Where $ \mathbf{X} = \left\lbrace \mathbf{x}_1,...,\mathbf{x}_N \right\rbrace  $, $ \mathbf{Z} = \left\lbrace \mathbf{z}_1,...,\mathbf{z}_N \right\rbrace  $ and $ \Theta = \left\lbrace \pi, \mathbf{A}, \phi \right\rbrace  $ is the models parameters. The parameter $ \pi $, is the initial parameter and \textbf{A} is the transition parameter, which holds the probabilities for transition between each hidden state.
The parameter $ \phi $, is the emission parameter. 

Given a set of data points the goal is to determine the model parameters in the HMM. This can be done by applying an estimation maximization algorithm on the following equation
\begin{equation}
Q(\Theta,\Theta^{\mathtt{old}})=
\sum_{\mathbf{z}}^{} p(\mathbf{X},\mathbf{Z}|\Theta^{\mathtt{old}})
\ln p(\mathbf{X},\mathbf{Z}|\Theta)
\end{equation}
             