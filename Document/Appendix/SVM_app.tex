\chapter{Support Vector Machines}
\section{Theory}

In this section of the appendix the soft margin support vector machine (SVM) is described. The SVM is a binary classifier, which is a problem with multi class. 
The solution for multiclass classification is using a one-vs.-one setup.
With only three classes this is achieved with relative ease. 
The integrated MATLAB statistic toolbox was used to process the data in the SVM case. 

SVM classify by making a decision boundary that maximizes the margin $ \gamma $.
The margin is the perpendicular distance from the decision boundary to the closest feature point.
The standard SVM have a linear decision boundary, which classifies by assigning the new data point to either class $ t=1 $ or $ t=-1 $. 
The decision function for a standard SVM for a test point $ x_{new} $ given by
\begin{equation}
t_{new} = \mathtt{sign}(\mathbf{w}^T \mathbf{x}_{new} +b)
\label{eq:SVM_lin}
\end{equation}
The parameter vector $ \mathbf{w} $ is found by maximizing the margin or minimizing the length of the parameter vector, because of the inverse relationship $ \gamma = \frac{1}{\|\mathbf{w}\|} $.
Moving one single data point can have a huge influence on the position of the decision boundary, The reason is in the constrain
\begin{equation}
t_n(\mathbf{w}^T \mathbf{x}_n + b) \geq 1
\end{equation}   

 
\section{Method}
Radial basis function (gauss med samme variance i alle retninger) 
\section{Results}

\begin{figure}[H]
\centering
\includegraphics{SVM_3_1digit_PCA10}
\caption{Results of using SVM classifiers and one digit spoken}
\label{fig:SVM3_1dig_PCA}
\end{figure}

\begin{table}[H]                                                    
\centering                                                          
\begin{tabular}{|l|c|c|c|c|}                                        
\hline                                                              
  & Speaker Jacob & Speaker Mose & Speaker Simon & Precision [\%] \\
\hline                                                              
Estimate Jacob & 3454.0 & 1408.0 & 1002.0 & 58.9 \\                 
\hline                                                              
Estimate Mose & 0.0 & 1746.0 & 12.0 & 99.3 \\                       
\hline                                                              
Estimate Simon & 0.0 & 300.0 & 2440.0 & 89.1 \\                     
\hline                                                              
Sensitivity [\%] & 100.0 & 50.6 & 70.6 & 73.7 \\                    
\hline                                                              
\end{tabular}                                                       
\caption{Confusion matrix - 1 digit}                                
\label{table:SVM_3_conf_1}                                          
\end{table} 


\section{Discussion}