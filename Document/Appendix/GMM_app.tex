\chapter{Gaussian Mixture Models}
\section{Theory}

Gaussian Mixture Models (GMM) is a way of finding and describing sub-populations in clusters of data points. 
It is done by fitting a specified number of Gaussian distributions to a population of data points.
Each distribution is a component of the model. 
The individual data points are then arranged into clusters based on which model component is most likely given the observed data point.

In this context the data points are features in a multidimensional feature space. Hence the distributions used for the GMM is multivariate Gaussians.

The clustering on basis of likelihood makes GMMs more robust than e.g. K-means, where data points are clustered similarly, but simply on the basis of Euclidean distance to the center of a model. 
This is because treating the components like Gaussians with means and variances instead spheres with uniform probability, gives a more nuanced picture of the strength of a data point’s relationship to a model component. 
It also gives rise to the notion of soft relations to clusters or data points related to more than one cluster.

When fitting the GMM the goal is to maximize the overall likelihood of the model for the entire population.
Firstly it is necessary to determine the number of distribution components the data should be fitted to. 
The number of distributions needed in a GMM greatly depends on the nature and origin(s)
of data.
This, however, does not mean that if a mixture model fits data well with a specific amount of distributions, then this is the actual number of sub-populations, or sources if you will.
Multiple sub-populations could be grouped together, or likewise sub-populations could be split, due to under/over-fitting.
For this reason experimentation is necessary to establish the optimal number of distributions needed to model data.

% What we did for this project



\subsection{The EM Algorithm}
The distributions of the mixture model are fitted to data by iteratively employing Expectation Maximization (EM).
This is done by either selecting an arbitrary guess of the means and variances of the model as a starting point or initializing the means with a few iterations of K-means, and setting the covariance matrices to the identity matrix. 
The latter is often used because even though GMM generally is more precise than K-means, it is also much more computationally heavy.
Therefore by initializing with K-means, a relatively light algorithm, the EM will converge on a good model faster.

A visual example of this convergence can be shown by applying GMM to dataset of eruption time vs. waiting time of the Old Faithful geyser in Yellowstone National Park, WY, USA.
The data has been normalized for simplicity. 
\\

\begin{figure}[H]
\centering
\includegraphics[width=0.25\textwidth]{GMM1}
\caption{Old Faithful data. GMM stared at an abitrary point}
\label{fig:GMM1}
\end{figure}

In Figure \ref{fig:GMM1} above the EM has just started from an arbitrary starting point.
Data shows two apparent clusters, so the model is generated with two components.
The data points are colored by association with the model components.
Note the fading in colors showing strength of relation to model component.
At this point the GMM does not fit data very well.

\begin{figure}[H]
\centering
\includegraphics[width=0.25\textwidth]{GMM2}
\caption{Old Faithful data. GMM after one iteration of EM}
\label{fig:GMM2}
\end{figure}

\begin{figure}[H]
\centering
\includegraphics[width=0.25\textwidth]{GMM3}
\caption{Old Faithful data. GMM after five iterations of EM}
\label{fig:GMM3}
\end{figure}

After five iterations the EM has honed in on the centers of the two clusters (Figure \ref{fig:GMM3}).
Note that K-means could have reached roughly this point in as many iterations, but at much less computational cost.

\begin{figure}[H]
\centering
\includegraphics[width=0.25\textwidth]{GMM4}
\caption{Old Faithful data. GMM after 20 iterations of EM}
\label{fig:GMM4}
\end{figure}

As seen in Figure \ref{fig:GMM4} above the model has converged on a well-fitting model after 20 iterations.
This means that further iterations would be superfluous, as they would yield only negligible increases in overall likelihood of the GMM. \\

The specific procedure for estimating GMM parameters using EM is as follows: % refREFERFEREFEREFEREFEREFEREF

\subsection*{Initialization:}

Choose initial estimates for model parameters $ w_{j}^{(0)}, \mu_{j}^{(0)}, \Sigma_{j}^{(0)} for j = 1, ... , k $.

\begin{itemize}

\item
$ k $ is the number of components.

\item
$ w_{j}^{(0)} $ is initial weight of the \textit{j}th component.

\item
$ \mu_{j}^{(0)} $ is initial mean of the \textit{j}th component.

\item
$ \Sigma_{j}^{(0)} $ is initial covariance matrix of the \textit{j}th component.

\end{itemize}




