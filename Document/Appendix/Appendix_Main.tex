\documentclass[]{report}


% Title Page
\title{TiNONS Project - Speaker Recognition: Appendix}
\author{Kasper Nielsen \& Alexander Rasborg Knuden}


\begin{document}
\maketitle

\chapter{Features}

When doing speaker recognition it is advantageous to classify on the basis of extracted features from speech data, rather than the audio samples themselves. \ref{1} % Reference

In this project, the features used for classification are based on Mel-frequency Cepstrum Coefficients (MFCC).
They have proven effective for use in speaker recognition, in other implementations.

% Noget om MFCC
% Basic overview
% udregning
% enkelte typer (Standard, delta, double-delta)
% Voicebox toolbox




\end{document}          
