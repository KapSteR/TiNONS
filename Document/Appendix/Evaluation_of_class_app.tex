\chapter{Evaluation of classification methods}
The individual classification methods are evaluated with datasets of varying complexities.
Most notably, with speakers uttering the same single digit ("\textit{ZERO}"), two different digits ("\textit{ZERO}" and "\textit{ONE}"), and ten different digits ("\textit{ZERO}", "\textit{ONE}" through "\textit{NINE}").


To evaluate the linear classifier, the confusion matrix is an ideal way of doing so.
The confusion matrix can show the classifications sensitivity, precision and accuracy for each class. 
The terms used to describe this is: false positive (FP), true positive (TP), false negative (FN) and true negative (TN).
The true terms are classes that are correctly classified and false terms are incorrectly classified.
The sensitivity is the probability of classifying the inputs as class $X$ for a input that are class $X$.
\begin{equation}
\mathtt{sensitivity}(X) = \dfrac{TP_X}{TP_X+\Sigma FN_X}
\label{eq:sensitivity}
\end{equation}
The precision is the probability for the estimated of class $X$ is correct.
\begin{equation}
\mathtt{precision}(X) = \dfrac{TP_X}{TP_X+\Sigma FP_X}
\label{eq:precision}
\end{equation}
The accuracy is the probability that the classification of any given class is correct, where N = total number of tests.
\begin{equation}
\mathtt{accuray}(X) = \dfrac{\Sigma TP_X}{\mathtt{N}}
\label{eq:accuracy}
\end{equation}