\subsection*{Gaussian Mixture Models}
Gaussian Mixture Models (GMM) is a way of finding and describing sub-populations in clusters of data points.
It is done by fitting a specified number of Gaussian distributions to a population of data points.
Each distribution is a component of the model. 
The individual data points are then arranged into clusters based on which model component is most likely given the observed data point.

The distributions of the mixture model are fitted to data by iteratively employing Expectation Maximization (EM).

\paragraph*{Training}
A GMM was trained for every speaker.
Then, for each speaker, a GMM is fitted to the respective speakers' training data, using MATLAB's Statistics Toolbox.

\paragraph*{Result}
The result of the GMM is a total accuracy of 94.6, 91.2 and 89.1 \% respectively for one, two and ten digits. 