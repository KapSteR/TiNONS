\subsection*{Linear Classification}

\begin{equation}
y(\mathbf{x}) = \mathbf{w}^T \mathbf{x}+w_0
\label{eq:lin_output}
\end{equation}

\paragraph*{Training}
The linear classifier is applied to the training dataset, containing feature vectors $(\mathbf{x}_n)$ and the target vectors $(\mathbf{t}_n)$.
The vectors are on the form:
\begin{equation}
\mathbf{\tilde{X}}=\left[ \begin{array}{c}\mathbf{x}_1^T \quad 1\\
\mathbf{x}_2^T \quad 1\\
...\\ 
\mathbf{x}_n^T \quad 1 \end{array} \right],
\;
\mathbf{T}=\left[ \begin{array}{c}
\mathbf{t}_1^T\\ 
\mathbf{t}_2^T\\ 
...\\
\mathbf{t}_n^T
\end{array} \right]
\label{eq:linearVectors}  
\end{equation} 

This calculation are done to determine the $\tilde{\mathbf{W}}$:
\begin{equation}
\tilde{\mathbf{W}} = \tilde{\mathbf{X}}^\dagger \mathbf{T} \approx  (\tilde{\mathbf{X}}^T \tilde{\mathbf{X}}+\mathbf{I})^{-1} \tilde{\mathbf{X}}^T\mathbf{T}
\label{eq:weightVector}  
\end{equation}

\paragraph*{Result}
The result of the linear classifier is a total accuracy of 55.9, 51.2 and 50.6 \% respectively for one, two and ten digits. 
